%Alvarez Lucas gains from Trade vs. gains from labor migration
\documentclass{article}
\usepackage{amsmath,amsfonts,amsthm,amssymb}
\usepackage{natbib}

\title{Comparing Further Gains from Trade to Gains from Labor Migration}
\author{David Jinkins}

\begin{document}
\maketitle

\section{Introduction}
In a recent article, Michael Clemens points out that although there are many papers calculating welfare gains from tariffs reduction, there are very few studies looking at gains from reducing barriers to labor migration--even though the latter gains are potentially much larger \citep{Clemens2011} . \citet{AlvarezLucas2007} is a recent study analysing gains from tariff reduction.  Using a modified Eaton-Kortum model to allow for tariff revenues to be passed back to consumers, the Alvarez-Lucas model implies a world aggregate welfare gain of about .5\% from reducing current tariffs to zero.  In this paper, I use the Alvarez-Lucas model to caculate gains to free labor migration.  The benchmark model implies welfare gains of about 8/%, or welfare gains 16 times larger than those from eliminating tariffs.\footnote{Although not directly comparable to welfare, it is instructive to note that .5/% of current world GDP is about 315 billion dollars, and 8/% is about 5 trillion dollars--about the size of Japan's GDP.      


\bibliographystyle{plainnat}
\bibliography{biglist}
\end{document}